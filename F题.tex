\documentclass{article}
\usepackage{ctex}
\usepackage{amsmath, amssymb, amsfonts}
\usepackage{geometry}

\geometry{a4paper, margin=1in}

\begin{document}

\section*{第 1 问数学建模(修正版 · 扩充稿)}

\subsection{集合与输入(与题面字段对齐,并行友好的规范化流程)}
设机组集合 $P$,航班集合 $F$。每名机组 $p \in P$ 具有基地 $\mathrm{Base}(p)$ 与三个二元资格标记:$\mathrm{Captain}(p) \in \{0,1\}$、$\mathrm{FirstOfficer}(p) \in \{0,1\}$、$\mathrm{Deadhead}(p) \in \{0,1\}$;这些字段将决定可用角色与是否允许乘机。每个航班 $f \in F$ 具有四元组属性 $\big(\mathrm{depStn}(f),\mathrm{arrStn}(f),\mathrm{depTime}(f),\mathrm{arrTime}(f)\big)$ 以及最低角色需求 $(\mathrm{req}_C(f),\mathrm{req}_F(f))$。所有时间以分钟为单位的绝对时间戳表达,以避免跨日、跨时区问题;机场使用离散代码并通过哈希字典一一映射到整数索引,便于稀疏结构的快速检索与并行化处理。为构造可连接弧,定义时间—机场兼容谓词
\begin{equation}
\mathsf{link}(f,g) \iff \big(\mathrm{arrStn}(f)=\mathrm{depStn}(g)\big) \land \big(\mathrm{depTime}(g)-\mathrm{arrTime}(f)\ge \mathrm{MinCT}\big),
\end{equation}
其中 $\mathrm{MinCT}=40$ 分钟由题面给定。仅当 $\mathsf{link}(f,g)$ 成立时加入弧 $(f \to g)$ 至集合 $A$。同时,为了让每名机组的行程图具备源汇结构,引入超级源 $S_p$ 与超级汇 $T_p$,并仅允许从基地可达的起飞航班与抵达基地的到达航班分别与 $S_p$ 和 $T_p$ 建立外接弧;此外置入零长弧 $(S_p \to T_p)$ 以承载“全期闲置”的边界情形。该图在时间维度上天然为有向无环(所有弧满足起点航班到达时间不晚于终点航班起飞时间),有利于后续单位流约束的数值稳定性与可解释性。
\par
上述规范化还包括:剔除 $\mathrm{depTime} \ge \mathrm{arrTime}$ 的异常记录(视为输入校验错误)、统一夏令时/跨日时间换算、按机场与起飞时间建立二级倒排索引以在 $O(\log n)$ 时间内检索候选后继。参数 $\mathrm{MaxDH}=5$ 定义每一航班可承载的乘机上限,仅在该航班实际起飞时有效(见 3.1)。

\subsection{决策变量(语义对齐与最小完备性)}
以最小完备性为原则设置变量族:覆盖二元 $z_f$ 表示航班 $f$ 是否满足最低配置且起飞;角色执行二元 $x^{C}_{p,f}, x^{F}_{p,f}$ 分别对应正机长与副机长的执飞;考虑“正机长可替副机长”的数据事实,引入 $x^{F\text{sub}}_{p,f}$ 捕捉该特例;乘机二元 $y_{p,f}$ 表示机组以乘客身份上机,用于空间转移但不贡献覆盖;节点使用二元 $u_{p,f}$ 将“该机组在该航班的任意形态出现”统一抽象出来,方便与弧流绑定;弧流二元 $w_{p,a}$ 则在可连接弧(含源/汇弧与零长弧)上刻画机组路径。上述变量足以在单一的线性整数规划中同时表达“最低配置—资格互斥—乘机容量—路径连续—基地闭合”五个核心方面,无需增设冗余层级。变量域均为 $\{0,1\}$,保证语义清晰与求解器友好。

\subsection{约束(含本次修正的关键处与完备性说明)}

\subsubsection{覆盖与“未起飞不配置”绑定(含乘机关断与容量)}
通过等式覆盖保证“满足最低配置 $\Leftrightarrow$ 航班起飞”,并用 y–z 绑定排除在未起飞航班上安排任何人(包括乘机)的可能性:
\begin{gather}
\sum_{p} x^{C}_{p,f} = \mathrm{req}_C(f)\,z_f, \quad
\sum_{p}\big(x^{F}_{p,f}+x^{F\text{sub}}_{p,f}\big) = \mathrm{req}_F(f)\,z_f, \\
\sum_{p} y_{p,f} \le \mathrm{MaxDH} \cdot z_f, \quad
y_{p,f} \le z_f \quad (\forall p,f).
\end{gather}
前两式将恰好满足最低需求与 $z_f=1$ 等价;后两式确保“乘机仅在已起飞航班上发生且不超容量”,完全等价于题面“未满足最低机组资格配置的航班不能配置任何机组人员”。该绑定对计量口径亦重要:后续所有统计(除未覆盖航班数)都以 $z_f=1$ 的样本集为分母。

\subsubsection{资格字段与互斥一致性(以数据驱动的可行域)}
令 $c_p=\mathbb{1}\{\mathrm{Captain}(p)=1\}, f_p=\mathbb{1}\{\mathrm{FirstOfficer}(p)=1\}, d_p=\mathbb{1}\{\mathrm{Deadhead}(p)=1\}$。以蕴含约束将变量域与字段绑定:
\begin{gather}
x^{C}_{p,f}\le c_p, \quad 
x^{F}_{p,f}\le f_p, \quad 
x^{F\text{sub}}_{p,f}\le c_p f_p, \quad 
y_{p,f}\le d_p, \\
x^{C}_{p,f}+x^{F}_{p,f}+x^{F\text{sub}}_{p,f}+y_{p,f}\le 1.
\end{gather}
其中 $x^{F\text{sub}}$ 的可行域严格为“既是 Captain 又拥有 FirstOfficer 替补资格”的正机长集合,杜绝“资格泛化”。互斥保证每人每航班至多一种参与形态,避免统计与成本的二重计费。

\subsubsection{单路径与基地闭合(单位流 + 节点—弧守恒)}
以单位流和节点—弧守恒排除“同人多条不联通链”的病态,确保每名机组的全期行程是一条从基地出发并最终回到基地的单路径:
\begin{gather}
u_{p,f}=x^{C}_{p,f}+x^{F}_{p,f}+x^{F\text{sub}}_{p,f}+y_{p,f}, \\
\sum_{a\in \delta^{-}(f)} w_{p,a}=u_{p,f}=\sum_{a\in \delta^{+}(f)} w_{p,a}, \\
\sum_{a\in \mathrm{out}(S_p)} w_{p,a}=1, \quad 
\sum_{a\in \mathrm{in}(T_p)} w_{p,a}=1.
\end{gather}
由于弧集 $A$ 仅包含“同机场且时间可连”的有向边,路径可达性天然蕴含“机场连续与最小连接时间”两条硬约束,源/汇弧进一步强制“基地起止”。零长弧 $(S_p \to T_p)$ 保证“完全闲置”的人员亦满足单位流。

\subsection{目标与字典序求解(严格按编号顺序与等式锁定)}
采用字典序多目标优化:层 ① 最大化覆盖 $\sum_f z_f$;得到最优 $Z^*$ 后加入等式 $\sum_f z_f=Z^*$ 锁定;层 ④ 在该最优面上最小化乘机次数 $\sum_{p,f}y_{p,f}$ 并锁定;层 ⑦ 在前两层最优面上最小化替补 $\sum_{p,f}x^{F\text{sub}}_{p,f}$。该流程可视为“分层求解 + 值锁定”的等价实现,避免通过极端权重近似字典序而引致的数值不稳定与尺度失衡。该做法严格契合题面“按编号次序”的目标执行逻辑。

\subsection{可连接弧的构造与复杂度控制(以机场—时间索引稀疏化)}
对每个机场建立按到达/起飞时间排序的事件表,仅在同机场相邻事件间差值 $\ge \mathrm{MinCT}$ 的候选之间连边,可将潜在的 $O(|F|^2)$ 连接压缩至近似线性规模。进一步地,对每一到达事件仅保留若干个“最早可行”后继以执行占优剪枝:若存在 $g_1, g_2$ 且 $\mathrm{depTime}(g_1) \le \mathrm{depTime}(g_2)$ 与其余属性一致,则 $(f \to g_2)$ 为支配劣势边,可删。源/汇弧的建立同理,按基地过滤。该稀疏化对 B 套大规模实例尤为关键,既保证“与规则一致”的可行性,又降低图边数与变量规模,维持 MILP 的可解性与可扩展性。

\subsection{结果口径与导出映射(与“第五部分”完全同口径)}
核心指标:满足/不满足机组配置航班数(由 $z_f$ 统计)、总体乘机次数($\sum y$)、替补使用次数($\sum x^{F\text{sub}}$)、程序运行分钟数。除“不满足数”外,其余统计均以 $z_f=1$ 样本集为分母,确保与题面口径一致。输出 UncoveredFlights.csv(记录 $z_f=0$ 的航班,含起降时刻与最低配置)与 CrewRosters.csv(按“机组—日期—段次”列出执行角色或乘机)。该映射与题面“结果指标与提交结果要求”一一对应,具有可复核性与可审计性。

\newpage

\section*{第 2 问数学建模(修正版 · 扩充稿)}

\subsection{集合与输入(引入自然日分割与跨日可达性)}
在第 1 问框架上引入自然日集合 $D$。定义分日映射 $\mathrm{depDate}(f)=\lfloor \mathrm{depTime}(f)/1440\rfloor$,据此将“日内执勤”限定为同一自然日起飞的航段序列(末段到达可跨日)。弧集拓展为两类:日内弧依旧遵循同机场与最小连接时间;跨日弧仅当“上一日末段到达机场 = 次日首段出发机场”且“次日首发—上一日末到 $\ge \mathrm{MinRest}$”时存在,用以表达执勤间休息下限。为后续“每日最多一条执勤”的计数与“执勤时长/飞行时长”的定义,需将每名机组在自然日 $d$ 的首发与末到时间作为连续变量并强制与日内已选节点绑定。参数采用题面给定数值:$\mathrm{MinCT}=40$、$\mathrm{MaxBlk}=600$、$\mathrm{MaxDP}=720$、$\mathrm{MinRest}=660$、$\mathrm{MaxDH}=5$。该分层化输入确保执勤层规则线性可表达且与数据字段一致。

\subsection{决策变量(新增执勤日层并保持与航段层对接)}
在第 1 问变量组的基础上,新增自然日维度:出勤指示 $D_{p,d} \in \{0,1\}$;当日首发/末到时间 $t^{\min}_{p,d}, t^{\max}_{p,d} \in \mathbb{R}_+$;当日飞行时长 $\mathrm{Blk}_{p,d}$ 与执勤时长 $\mathrm{DP}_{p,d}$(二者口径不同,见 3.5);为执勤平衡目标准备的全期执勤总时长偏差 $\eta_p^{\pm} \ge 0$。这些变量通过“日内节点选择聚合 + 大M 线性化”与航段层变量紧耦合,使执勤定义成为约束导出量,避免引入新的离散对象而致模型不稳。

\subsection{约束(执勤层规则的严格等式与跨日衔接)}

\subsubsection{覆盖与乘机关断(同第 1 问)}
与第 1 问相同,保持等式覆盖与 y–z 绑定与乘机容量关断,不再赘述。该约束在第二问的可行域内重新求最优覆盖,而非沿用第一问最优值,确保在新硬约束的约束面上具有可行最优解。

\subsubsection{资格与互斥(字段驱动的一致性)}
完全沿用第一问的字段联动与互斥结构,保证执勤层的新增约束不会引入资格歧义。

\subsubsection{单路径—基地闭合(单位流保障一日一链的上位结构)}
与第 1 问相同的单位流与节点—弧守恒建立全期的单链结构,从全局上排除“同日出现两段不相连链”的可能;其日内具体表现由 3.4 的“开始块计数”约束作细化控制。

\subsubsection{“每日最多一条执勤”的块计数与二阶段绑定}
先用出勤指示将“是否当日参与任何航班(含乘机)”与 $D_{p,d}$ 绑定,再以开始块思想限制计数不超过 1:把“无来自同日前序入弧的已选节点”识别为该日的执勤起点,其数量不超过 1。该构造优于单纯的“按日聚合”约束,因为它消除了“同日两段被一条跨日弧隔开”的隐患,保证结构唯一性。

\subsubsection{执勤飞行与执勤总时长:严格等式、上限与口径}
以定义等式刻画两类时长:$\mathrm{Blk}_{p,d}$ 由当日所有被执行航段的飞行时间(起降差)求和,不含乘机;$\mathrm{DP}_{p,d}=t^{\max}_{p,d}-t^{\min}_{p,d}$ 则代表“首发至末到的跨度”,其中间连接与乘机时间均计入执勤。用大M 选择器把 $t^{\min}, t^{\max}$ 与当日所选节点绑定。上限约束
\begin{equation}
\mathrm{Blk}_{p,d} \le \mathrm{MaxBlk} \cdot D_{p,d}, \quad \mathrm{DP}_{p,d} \le \mathrm{MaxDP} \cdot D_{p,d}
\end{equation}
直接对应题面参数。将口径处理为严格等式而非近似,是第二问与题面一致性的关键。

\subsubsection{跨日衔接:机场一致与休息下限}
(i)图构造层面,只对“上一日末到机场 = 次日首发机场”且“休息 $\ge \mathrm{MinRest}$”的节点对建立跨日弧;(ii)变量层面,对连续出勤日强制 $t^{\min}_{p,d+1}-t^{\max}_{p,d}\ge \mathrm{MinRest}$。两层一致的表达既避免不必要的弧,也提升了模型可解释性。

\subsection{目标与字典序求解(第二问可行域内逐层锁定)}
按题面编号推进:① 覆盖最大化得到 $Z^{(2)}_*$ 并锁定;② 执勤成本最小化(按人按日以 $\mathrm{DutyCostPerHr} \times \mathrm{DP}$ 计)得到 $C^{(2)}_*$ 并锁定;④ 乘机最少 得到 $Y^{(2)}_*$ 并锁定;⑤ 执勤平衡以 $L_1$ 偏差最小化($\sum_p(\eta_p^{+}+\eta_p^{-})$)——选择 $L_1$ 而非二次项的理由在于可保持线性结构与鲁棒性;⑦ 替补最少在最末层压缩。该顺序严格遵循题面“依编号次序”,且每层等式锁定以免低序目标破坏高序目标。

\subsection{图构建与规模控制(与规则一致的稀疏化与窗口化)}
日内沿最小连接时间剪枝,跨日沿休息下限与机场一致性剪枝;进一步在大型实例上使用时间窗口:将计划期按天或周切分为多个相邻窗口,在窗口内实例化变量与弧并求解,边界处通过“前窗锁定的高序目标值 + 仅保留必要跨窗弧”的策略维持全局一致性。这种工程化处理兼顾了题面规则与大规模可解性。

\subsection{结果指标与导出(第二问新增口径的可复核实现)}
除第一问四项外,新增:一次执勤飞行时长与执勤时长的最小/平均/最大,机组人员执勤天数的最小/平均/最大,总体执勤成本。这些统计可直接从 $\mathrm{Blk}_{p,d}, \mathrm{DP}_{p,d}, D_{p,d}$ 与 $\mathrm{DutyCostPerHr}$ 聚合得到;口径仍以 $z_f=1$ 样本集为准。输出文件 CrewRosters.csv 建议增补 $t^{\min}_{p,d}, t^{\max}_{p,d}, \mathrm{Blk}_{p,d}, \mathrm{DP}_{p,d}, D_{p,d}$ 以便外部复核;UncoveredFlights.csv 维持第一问结构。

\newpage

\section*{第 3 问数学建模(修正版 · 扩充稿)}

\subsection{集合与输入(任务环—周期扩展与候选生成原则)}
在第二问的航段—执勤双层基础上引入“任务环—排班周期”层。为兼顾线性表达与规模可控,对每名机组 $p$ 预生成有限集合 $\mathcal{R}_p$ 的任务环候选:每个候选 $r$ 是由若干按时间递增排列的执勤日序列及其间休息日组成,满足(i)首日从 $\mathrm{Base}(p)$ 起飞;(ii)末日抵达 $\mathrm{Base}(p)$;(iii)相邻执勤之间满足休息下限 $\mathrm{MinRest}$;(iv)任意滑动窗口内的连续执勤天数不超过 $\mathrm{MaxSuccOn}$;(v)跨日空间一致(上一执勤终到机场 = 下一执勤起点机场)。为与周期性指标对接,记录每一候选的在外总时长 $\mathrm{TAFB}_r$ 及其在日维度上的覆盖 $B_{r,d}$ 与起止标记 $S_{r,d}, E_{r,d}$。候选可以通过“以执勤图为状态图的资源约束最短路(RCSP)扩展”生成:以“时间、基地触达、连续执勤计数”为资源,遵循休息与空间一致性累积约束,按成本或启发式评分产生有限条可行环,以控制 $|\mathcal{R}_p|$ 的规模。周期边界采用延展窗口处理,使跨界环亦可按基地一致性在相邻周期中闭合。

\subsection{决策变量(三层联动的最小变量化表达)}
在保留第二问的 $z, x^{\cdot}, y, u, w, D, t^{\min/\max}, \mathrm{Blk}, \mathrm{DP}$ 变量的同时,新增环选择二元 $s_{p,r} \in \{0,1\}$,以及由候选聚合而成的环起止标记 $\mathrm{Start}_{p,d}, \mathrm{End}_{p,d}$(它们由 $S_{r,d}, E_{r,d}$ 与 $s_{p,r}$ 的线性组合定义)。为 ⑥ 层的环平衡目标设置偏差 $\xi_p^{\pm} \ge 0$,并定义个体在外总时长 $\mathrm{TAFB}^{\mathrm{tot}}_{p}=\sum_{r \in \mathcal{R}_p}\mathrm{TAFB}_r s_{p,r}$。变量族保持最小完备性:执勤与环均不额外引入序列位置变量,而是通过“候选—日覆盖—起止聚合”与“航段—执勤—环”的覆盖/守恒关系线性耦合,避免二次离散化造成的指数级爆炸。

\subsection{约束(环—日—航段三层一致性与周期规则)}

\subsubsection{航段覆盖、乘机容量与资格互斥(同第二问)}
保持 2.问的三组约束不变:等式覆盖与 y–z 绑定、资格字段联动、同航段互斥;其可行域在第三问中仍为下层约束的基础。

\subsubsection{单路径—基地闭合与可连性(同第二问)}
全期单位流 + 节点—弧守恒保持“唯一单链”,与候选层的“基地起止”共同形成强一致性:若某日为环起点,则在航段层必有一条从基地起飞的已选节点;若为环终点,则必有一条到基地的已选节点。

\subsubsection{执勤层的等式口径与跨日休息(同第二问)}
继续以定义等式表达 $\mathrm{Blk}_{p,d}, \mathrm{DP}_{p,d}$ 与 $t^{\min/\max}$ 的关系,并保持 $\mathrm{MaxBlk}$、$\mathrm{MaxDP}$ 与 $\mathrm{MinRest}$ 的上限与下限,确保环内部的每一对相邻执勤在“时间—空间”两维上均可达。

\subsubsection{环选择与日覆盖一致性(环—日耦合的线性化)}
对每名机组与每一日,令
\begin{equation}
\mathrm{Away}_{p,d}:=\sum_{r\in\mathcal{R}_p} B_{r,d}\,s_{p,r}.
\end{equation}
用 $\mathrm{Away}_{p,d} \ge D_{p,d}$ 强化“凡出勤之日必处于某个已选环之内”;并以
\begin{equation}
\mathrm{Start}_{p,d}=\sum_{r} S_{r,d}\,s_{p,r}, \quad 
\mathrm{End}_{p,d}=\sum_{r} E_{r,d}\,s_{p,r}
\end{equation}
聚合起止日。随后通过
\begin{gather}
\mathrm{Start}_{p,d} \le \sum_{f:\,\mathrm{depDate}(f)=d,\,\mathrm{depStn}(f)=\mathrm{Base}(p)} u_{p,f}, \\
\mathrm{End}_{p,d} \le \sum_{f:\,\mathrm{depDate}(f)=d,\,\mathrm{arrStn}(f)=\mathrm{Base}(p)} u_{p,f}
\end{gather}
把环边界与“基地触达”在航段层对齐,实现跨层一致的基地起止。

\subsubsection{周期规则:MaxTAFB、MinVacDay、MaxSuccOn(自然日口径)}
\begin{enumerate}
    \item[(i)] 在外总时长上限:$\sum_{r}\mathrm{TAFB}_r s_{p,r} \le \mathrm{MaxTAFB}$。
    \item[(ii)] 环间最小休假天数:若 $d$ 为某环结束日,则接下来的 $\mathrm{MinVacDay}$ 个自然日不得出勤
    \begin{equation}
    \sum_{i=1}^{\mathrm{MinVacDay}} D_{p,d+i} \le \mathrm{MinVacDay}\,(1-\mathrm{End}_{p,d}).
    \end{equation}
    \item[(iii)] 连续执勤天数上限:对任意长度为 $\mathrm{MaxSuccOn}+1$ 的滑动窗口 $W$,约束 $\sum_{d\in W} D_{p,d} \le \mathrm{MaxSuccOn}$。
\end{enumerate}
三者共同塑造“周期节奏”,其量纲分别是分钟与自然日;用日口径计数可避免以分钟近似“天数”带来的舍入偏差。

\subsection{目标与字典序求解(第三问可行域内的全局顺序)}
严格执行全题一致的字典序:① 覆盖最大化(得 $Z^{(3)}_*$ 并锁定);② 执勤成本最小(得 $C^{(3)}_*$ 并锁定);③ 任务环成本最小($\sum_{p,r}\mathrm{ParingCostPerHr}(p) \cdot \mathrm{TAFB}_r s_{p,r}$)并锁定;④ 乘机最少并锁定;⑤ 执勤平衡($L_1$ 偏差);⑥ 任务环平衡(以 $\mathrm{TAFB}^{\mathrm{tot}}_{p}$ 的 $L_1$ 偏差最小化);⑦ 替补最少。特别强调:③ 的任务环成本不包含执勤成本,避免重复计费;② 与 ③ 顺序不可互换,以符合题面编号。每层采用“求最优—等式锁定”的过程,等价于严格的字典序最优化。

\subsection{规模与实现要点(候选生成 + 主问题的工程化落地)}
候选生成:以“执勤日图”为状态空间,状态包含“当前日期、当前位置、连续执勤计数、是否在外、基地标志”,转移受“机场一致 + $\mathrm{MinRest}$”与“连续执勤上限”的资源约束;采用标号算法或双界限的 RCSP 生成若干成本占优的环候选,并在生成阶段即计算 $\mathrm{TAFB}_r、B_{r,d},S_{r,d},E_{r,d}$。通过剪枝规则(如起止日对称性、劣势支配)控制每人候选数量。主问题与第二问共享大部分变量与约束,仅新增环选择与周期规则,保持稀疏性与线性结构。对于 B 套实例,建议在周窗口内滚动:先锁定窗口内的①②③最优值,再将窗口边界的“起止日选择”以割的形式传递给下一窗口,确保全周期一致。

\subsection{结果指标与导出(第三问新增口径)}
除前两问指标外,新增:一/二/三/四天任务环的数量分布与总体任务环成本(万元)。任务环数量分布可由 $\mathrm{End}_{p,d}-\mathrm{Start}_{p,d}$ 的日跨度统计得到;任务环成本由 $\sum_{p,r}\mathrm{ParingCostPerHr}(p) \cdot \mathrm{TAFB}_r s_{p,r}$ 聚合给出。输出 CrewRosters.csv 建议增补“任务环编号、起止日期、$\mathrm{TAFB}_r$ 与是否跨周期”的字段以便复核;UncoveredFlights.csv 继续作为结构性缺口的证据链。所有统计(除未覆盖航班数)仍以 $z_f=1$ 样本集为准。

\section*{总述}
以上加长稿在不改变先前“修正版”思想与结构的前提下,系统加密了数据规范化、变量语义、约束完备性、字典序求解实施、弧/候选构造与规模控制、指标映射等关键环节,确保三问从“航段—执勤—任务环—周期”的层层递进在同一线性优化范式内完成,且严格与题面条款、参数与结果口径一一对齐。

\end{document} 